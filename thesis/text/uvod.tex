\chapter {Úvod}
Používanie desktopových a webových aplikácií či informačných systémov v medicíne sa stáva neoddeliteľnou súčasťou moderného sveta. Informatika v tejto oblasti zachraňuje životy pacientov, čo bolo možné vidieť najmä počas celosvetovej pandémie COVID-19, ktorá určitým spôsobom zasiahla každého z nás.

V súčasnosti je informatika nápomocná aj pri analýze pohybu myokardu pri podozrení na ischemickú chorobu srdca. Toto podozrenie je možné overiť snímkovaním srdca pomocou magnetickej rezonancie a aplikovaním tzv. SPAMM techniky počas tohto snímkovania. Táto technológia spôsobí vykreslenie mriežky na snímkach z magnetickej rezonancie, ktorá obalí srdcový sval a deformuje sa s pohybom myokardu. Následne je nutné vygenerovať mriežku, ktorá sa zarovná s vygenerovanou SPAMM mriežkou na každej snímke zobrazujúcej časť srdcového cyklu. Analýzou pohybu tejto mriežky je možné dospieť k záveru, či sú nutné ďalšie medicínske procedúry týkajúce sa zdravia snímkovaného pacienta.

Pre tieto účely bola vytvorená desktopová aplikácia, ktorej účelom je táto analýza pohybu myokardu. Nevýhodou tejto aplikácie je jej nedostupnosť na najviac používaných operačných systémoch, nakoľko túto aplikáciu je možné spustiť len v linuxovom prostredí. Taktiež je jej inštalácia zdĺhavá a náročná, čo len znemožňuje prístup lekárom v tomto odbore pracovať s touto aplikáciou.

Cieľom tejto práce je implementovať webovú aplikáciu podobnú súčasnej aplikácii za účelom vyriešenia problémov s dostupnosťou súčasnej aplikácie. Tento krok by umožnil posunúť aplikáciu bližšie k odborníkom zaoberajúcimi sa analýzou pohybu srdcového svalu.

Rešeršná časť sa zaoberá najprv princípom snímkovania magnetickej rezonancie a uvedenou SPAMM technikou. Následne sa práca venuje analýze súčasnej aplikácie, čo zahŕňa popis jej účelu, použitých technológií a pomocných podprogramov. Po tejto analýze bola súčasná aplikácia zostavená a spustená, za účelom popisu jej používateľského rozhrania a testovania funkcionality. Testovaním súčasnej aplikácie bolo zistený problém s algoritmom zodpovedným za zarovnanie používateľom vytvorenej mriežky s mriežkou vytvorenou SPAMM technikou.

Diplomová práca ďalej pokračuje analýzou a návrhom novej webovej aplikácie, od analýzy štruktúry mriežky a jej parametrov, cez požiadavky kladené na túto aplikáciu po popis prípadov použitia, ktoré nadväzujú na zozbierane funkčné a nefunkčné požiadavky. Technológie pre vývoj webovej aplikácie sú taktiež súčasťou analýzy webovej aplikácie spolu s porovnaním možných architektúr webovej aplikácie. Na základe tohto porovnania sa vyberie výsledná architektúra aplikácie. Keďže sa aplikácia nebude budovať úplne od začiatkov, bolo taktiež potrebné zanalyzovať vhodné frameworky pre jej tvorbu. Táto kapitola ďalej pokračuje analýzou spracovania MR snímiek vo webovom prehliadači a skúmaním dostupných možností implementácie generovanej mriežky. Koniec kapitoly sa venuje návrhom používateľského rozhrania aplikácie spolu s návrhom komunikácie webového rozhrania so serverom za účelom výpočtu zarovnania používateľom vytvorených mriežok s mriežkami vytvorenými SPAMM technikou.

Implementačná časť práce sa venuje počiatočnej štruktúre projektu so zoznamom použitých balíčkov, na základe ktorých je implementovaná rozličná funkcionalita v rámci webovej aplikácie. Najväčšia pozornosť je venovaná samotnej implementácii aplikácie -- a to najmä implementácii zobrazenia a úprave parametrov používateľom vytvorenej mriežky. Pre účely testovania aplikácie bola aplikácia taktiež nasadená pomocou služby Vercel.

Testovanie aplikácie odhalilo pár chýb v implementácii webovej aplikácie. Tieto chyby boli následne opravené a ich príčina detailne popísaná v kapitole veonvaná testovaniu webovej aplikácie.

Nakoľko sa nepodarilo implementovať SPAMM algoritmus z dôvodu nekompletnej implementácie tohto algoritmu v pôvodnej aplikácii, boli autorom pokračujúcim na vytvorenej webovej aplikácii zanechané odporúčania pre jej ďalší rozvoj.

