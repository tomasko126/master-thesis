\department{Katedra softwarového inženýrství}
\title{Webové rozhranie pre spracovanie SPAMM tagovaných dát z magnetickej rezonancie}
\authorGN{Tomáš} 
\authorFN{Taro}
\authorWithDegrees{Bc. Tomáš Taro} 
\author{Tomáš Taro}
\supervisor{Ing. Petr Pauš, Ph.D.}
\acknowledgements{Touto cestou by som sa chcel poďakovať Ing. Petrovi Paušovi Ph.D. za vedenie a trpezlivosť pri vypracovávaní tejto diplomovej práce. Obrovské ĎAKUJEM patrí prevažne mojej rodine a priateľke za neustálu podporu, najmä počas posledného roku, ktorý bol psychicky náročný. Taktiež by som sa chcel poďakovať testerom aplikácie, a to menovite Erikovi Ekemu, Timei Bartalskej a Ing. Radkovi Galabovi. Touto cestou by som sa chcel poďakovať do hudobného neba Mekymu Žbirkovi, ktorého jedinečná hudba a svojrázny hlas bol tým potrebným spríjemnením času počas písania tejto diplomovej práce.}
\abstractCS{Cieľom tejto diplomovej práce je vytvoriť webovú aplikáciu, ktorej úlohou bude analýza pohybu myokardu z DICOM snímiek magnetickej rezonancie.

Spočiatku sa diplomová práca venuje analýze súčasnej desktopovej aplikácie \uv{DICOM Viewer}. Nasleduje analýza webovej aplikácie, ktorá zahŕňa analýzu požiadaviek a k nim príslušných prípadov použitia, technológií pre vývoj aplikácie, jej architektúry, či prepojenia webovej aplikácie so súčasnou aplikáciou. Taktiež bolo nutné zistiť, akým spôsobom sa budú spracovávať importované DICOM snímky, či implementovať mriežka zobrazená nad týmito snímkami.

Na základe poznatkov z horeuvedených sekcií je navrhnuté používateľské rozhranie a komunikácia so serverom. Nasledujúca kapitola sa venuje implementácii webovej aplikácie a jej nasadenia. Následne bola správnosť funkcionality výslednej webovej aplikácie otestovaná.

Nakoľko je pre plné využitie aplikácie nutné doimplementovať potrebnú funkcionalitu, boli pre autorov pokračujúcich vo vývoji tejto aplikácie zanechané odporúčania týkajúce sa jej následného vývoja.
}
\abstractEN{The goal of this diploma thesis is to create a web application, the task of which will be to analyze the movement of the myocardium from DICOM magnetic resonance images.

Initially, the thesis deals with the analysis of the current \uv{DICOM Viewer} desktop application. The following is an analysis of the web application, which includes an analysis of the requirements and their corresponding use cases, technologies for the development of the application, its architecture and the connection of the web application with the current application. It was also necessary to find out how the imported DICOM images will be processed or how the grid displayed above these images will be implemented.

Based on the knowledge from the above sections, the user interface and communication with the server is designed. The following chapter deals with the implementation of the web application and its deployment. Subsequently, the correctness of the functionality of the resulting web application was tested.

As it is necessary to implement additional functionality for the full use of the application, recommendations regarding the subsequent development of this application were left for the authors continuing the development of this application.
\clearpage
}
\placeForDeclarationOfAuthenticity{V~Prahe}
\declarationOfAuthenticityOption{4} 
\keywordsCS{analýza myokardu, Cornerstone, DICOM, magnetická rezonancia, SPAMM}
\keywordsEN{myocardial analysis, Cornerstone, DICOM, magnetic resonance, SPAMM}